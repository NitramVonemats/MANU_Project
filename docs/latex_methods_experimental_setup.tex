% ==============================================================================
% EXPERIMENTAL SETUP - Methods Section
% Ready to paste into IEEE/ACM format paper
% Generated: 2026-01-19
% ==============================================================================

\subsection{Experimental Setup}
\label{sec:experimental_setup}

% ------------------------------------------------------------------------------
% Data Splits
% ------------------------------------------------------------------------------
\textbf{Data Splits and Preprocessing.}
All datasets were obtained from the Therapeutics Data Commons (TDC)~\cite{huang2021therapeutics}
and split into training (80\%), validation (10\%), and test (10\%) sets using a fixed
random seed (\texttt{seed=42}) for reproducibility. The same data splits were used
consistently across all model types (GNN, ChemBERTa, Morgan Fingerprints) to ensure
fair comparison. SMILES strings were canonicalized and converted to molecular graphs
with 19 node features representing physicochemical properties (TPSA, LogP, molecular
weight, hydrogen bond donors/acceptors, aromatic rings, rotatable bonds, and fraction
of sp$^3$ carbons). Edge features encoded bond types (single, double, triple, aromatic).

% ------------------------------------------------------------------------------
% Training Protocol
% ------------------------------------------------------------------------------
\textbf{Training Protocol.}
All models were trained exclusively on the training set (80\% of data). The validation
set (10\%) was reserved for model selection, early stopping, and hyperparameter
optimization metric evaluation. The test set (10\%) remained completely held out during
all training and hyperparameter tuning procedures to prevent data leakage. Importantly,
\emph{no models were retrained on combined train+validation sets after hyperparameter
selection}, ensuring a consistent training budget (80\% of data) across all approaches
for fair comparison.

% ------------------------------------------------------------------------------
% Hyperparameter Optimization
% ------------------------------------------------------------------------------
\textbf{Hyperparameter Optimization.}
For GNN models, we performed systematic hyperparameter optimization (HPO) using six
metaheuristic algorithms:
\begin{itemize}[noitemsep,topsep=0pt]
    \item \textbf{Random Search} (baseline): Uniform sampling from hyperparameter space
    \item \textbf{Particle Swarm Optimization (PSO)}~\cite{kennedy1995particle}:
          Population-based swarm intelligence
    \item \textbf{Artificial Bee Colony (ABC)}~\cite{karaboga2007artificial}:
          Bee foraging behavior simulation
    \item \textbf{Genetic Algorithm (GA)}~\cite{holland1992genetic}:
          Evolutionary selection and crossover
    \item \textbf{Simulated Annealing (SA)}~\cite{kirkpatrick1983simulated}:
          Thermodynamic annealing process
    \item \textbf{Hill Climbing (HC)}: Greedy local search
\end{itemize}

Each algorithm was run for 10 trials, yielding 60 total hyperparameter evaluations per
dataset (6 algorithms $\times$ 10 trials). Hyperparameters optimized included:
learning rate ($10^{-4}$ to $10^{-2}$), number of graph layers (2--4), hidden channels
(64--256), dropout rate (0.0--0.5), and GNN architecture type (GAT, GCN, GIN).
Validation F1 score (classification tasks) or RMSE (regression tasks) guided the
optimization process. All implementations used NiaPy~\cite{vrbancic2018niapy}
optimization library with population size 5 and maximum 10 iterations per trial.

% ------------------------------------------------------------------------------
% Early Stopping
% ------------------------------------------------------------------------------
\textbf{Early Stopping and Regularization.}
All GNN models employed early stopping with patience of 10 epochs to prevent overfitting.
Training terminated if validation performance did not improve for 10 consecutive epochs.
Dropout regularization~\cite{srivastava2014dropout} was applied after each graph
convolution and fully connected layer. Maximum training duration was limited to 50 epochs.

% ------------------------------------------------------------------------------
% Foundation Model Baselines
% ------------------------------------------------------------------------------
\textbf{Foundation Model Baselines.}
We compared GNN performance against two baseline approaches:

\begin{enumerate}[noitemsep,topsep=0pt]
    \item \textbf{ChemBERTa}~\cite{chithrananda2020chemberta}: A BERT-based transformer
          model pretrained on 77 million molecules from the ZINC15 database. Due to
          computational constraints (CPU-only training), ChemBERTa was used as a
          \emph{frozen feature extractor} without task-specific fine-tuning. Molecular
          embeddings (768-dimensional) were extracted and fed to a 2-layer MLP
          downstream predictor (hidden dimensions: 256, 128) trained via scikit-learn's
          MLPClassifier/MLPRegressor with Adam optimizer, learning rate $10^{-3}$, and
          early stopping (patience=10 epochs).

    \item \textbf{Morgan Fingerprints (ECFP4)}: Classical circular fingerprints with
          radius=2 and 2048 bits, serving as a strong non-learned baseline. Downstream
          predictors were Random Forest classifiers/regressors with 100 trees,
          max\_depth=10, and default scikit-learn~\cite{pedregosa2011scikit} parameters.
\end{enumerate}

Therefore, our comparison evaluates GNN models with systematic HPO against pretrained
foundation models in feature-extraction mode (ChemBERTa) and classical baselines
(Morgan Fingerprints), both without hyperparameter optimization.

% ------------------------------------------------------------------------------
% Evaluation Metrics
% ------------------------------------------------------------------------------
\textbf{Evaluation Metrics.}
For regression tasks (ADME properties: Caco-2 permeability, half-life, hepatocyte
clearance, microsomal clearance), we report:
\begin{itemize}[noitemsep,topsep=0pt]
    \item \textbf{Root Mean Squared Error (RMSE)}: $\sqrt{\frac{1}{n}\sum_{i=1}^{n}(y_i - \hat{y}_i)^2}$
    \item \textbf{Mean Absolute Error (MAE)}: $\frac{1}{n}\sum_{i=1}^{n}|y_i - \hat{y}_i|$
    \item \textbf{Coefficient of Determination (R²)}: $1 - \frac{\sum(y_i - \hat{y}_i)^2}{\sum(y_i - \bar{y})^2}$
\end{itemize}

For classification tasks (toxicity prediction: hERG cardiotoxicity, Tox21 nuclear
receptor assays), we report:
\begin{itemize}[noitemsep,topsep=0pt]
    \item \textbf{F1 Score}: Harmonic mean of precision and recall:
          $2 \cdot \frac{\text{precision} \cdot \text{recall}}{\text{precision} + \text{recall}}$
    \item \textbf{Area Under ROC Curve (AUC-ROC)}: Probability that a randomly chosen
          positive example ranks higher than a negative example
    \item \textbf{Accuracy}: Fraction of correctly classified examples
\end{itemize}

All metrics are computed on the held-out test set. For Tox21 (which contains 12 nuclear
receptor assays), we evaluate the NR-AR (Nuclear Receptor - Androgen Receptor) assay
as a representative single-task binary classification problem. Multi-task learning
across all 12 Tox21 assays is left for future work.

% ------------------------------------------------------------------------------
% Implementation Details
% ------------------------------------------------------------------------------
\textbf{Implementation Details.}
GNN models were implemented using PyTorch 2.0~\cite{paszke2019pytorch} and PyTorch
Geometric 2.3~\cite{fey2019pyg}. Three GNN architectures were evaluated:
\begin{itemize}[noitemsep,topsep=0pt]
    \item \textbf{Graph Attention Network (GAT)}~\cite{velivckovic2018gat}:
          Multi-head attention mechanism with learnable attention weights
    \item \textbf{Graph Convolutional Network (GCN)}~\cite{kipf2017gcn}:
          Spectral graph convolutions with symmetric normalization
    \item \textbf{Graph Isomorphism Network (GIN)}~\cite{xu2019gin}:
          Sum aggregation with learnable MLP for maximum expressive power
\end{itemize}

Models were trained with Adam optimizer~\cite{kingma2015adam}, batch size 32, and
gradient clipping (max norm=1.0). For regression tasks, mean squared error (MSE) loss
was minimized; for classification, binary cross-entropy (BCE) loss. All experiments
were conducted on CPU (Intel Core i7-8700K, 32GB RAM) due to resource constraints.
Training time per model ranged from 2--15 minutes depending on dataset size and
hyperparameters.

% ------------------------------------------------------------------------------
% Reproducibility
% ------------------------------------------------------------------------------
\textbf{Reproducibility.}
All experiments used fixed random seed (\texttt{seed=42}) and deterministic operations
where possible (PyTorch deterministic algorithms, fixed CUDA convolution benchmarking).
Complete source code, hyperparameter configurations, trained model checkpoints, and
experimental results are publicly available at:
\url{https://github.com/[USERNAME]/MANU\_Project}.
